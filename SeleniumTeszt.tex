\documentclass[12pt]{article}
\usepackage[utf8]{inputenc}
\usepackage{amsmath}
\usepackage{listings}
\usepackage{xcolor}

\title{A Selenium Alapú Automatizált Tesztelés Részletes Bemutatása}
\author{Pusztai Gábor, Iváncsik Ábrahám, Fehér Dominik}

\lstset{
	backgroundcolor=\color{lightgray},
	basicstyle=\ttfamily\footnotesize,
	breaklines=true,      
	captionpos=b,
	numbers=left,
	numberstyle=\tiny\color{gray},
	stepnumber=1,
	numbersep=5pt,
	tabsize=4,
	showspaces=false,
	showstringspaces=false,
	showtabs=false,
	keywordstyle=\color{blue},
	commentstyle=\color{green},
	stringstyle=\color{red}
}

\begin{document}
	
	\maketitle
	
	\section{Bevezetés}
	
	A Selenium egy nyílt forráskódú keretrendszer, amelyet a webalkalmazások automatizált tesztelésére használnak. A Python programozási nyelv és a Selenium WebDriver segítségével szimulálhatunk felhasználói interakciókat, például gombnyomásokat, szövegbevitel, űrlapkitöltéseket, valamint képernyőképeket is készíthetünk a tesztelés során.
	
	Ebben a dokumentumban egy egyszerű Selenium tesztet elemezünk, amely automatizálja egy webalkalmazás bejelentkezési folyamatát. A célunk az, hogy bemutassuk a Selenium használatának alapvető lépéseit, valamint megmagyarázzuk, hogyan használhatjuk ezt a keretrendszert automatizált teszteléshez.
	
	\section{A Kód Részletes Bemutatása}
	
	A következő Python kód egy egyszerű Selenium tesztet hajt végre, amely a bejelentkezési folyamatot automatizálja egy weboldalon. A kód folyamatosan végrehajt egy sor műveletet: beírja az email címet és jelszót, rögzíti a képerőképeket, majd bejelentkezik a rendszerbe.
	
	\begin{lstlisting}[language=Python, caption=Az Automatizált Tesztelési Kód]
		from selenium import webdriver
		from selenium.webdriver.common.keys import Keys
		from selenium.webdriver.common.by import By
		import time
		
		options = webdriver.ChromeOptions()
		driver = webdriver.Chrome(options=options)
		driver.set_window_size(1920, 1080)
		driver.get("https://esemenyrendezo.onrender.com/login")
		driver.find_element(By.XPATH, "//input[@type='email']").send_keys('TothB')
		driver.save_screenshot("screenshot1.png")
		time.sleep(2)
		driver.find_element(By.XPATH, "//input[@type='password']").send_keys('a')
		driver.save_screenshot("screenshot2.png")
		time.sleep(2)
		driver.find_element(By.CLASS_NAME, "login-btn").click()
		time.sleep(2)
		driver.save_screenshot("screenshot3.png")
		time.sleep(5)
		driver.quit()
	\end{lstlisting}
	
	\subsection{A Kód Magyarázata}
	
	Most nézzük meg a kód egyes részeit, és magyarázzuk el, hogyan működik.
	
	\subsubsection{Selenium WebDriver inicializálása}
	
	Az első két sorban a szükséges Selenium csomagokat importáljuk:
	
	\begin{lstlisting}
		from selenium import webdriver
		from selenium.webdriver.common.keys import Keys
		from selenium.webdriver.common.by import By
	\end{lstlisting}
	
	- A \texttt{webdriver} a Selenium alapvető komponense, amely a böngészők vezérlésére szolgál.
	- A \texttt{Keys} osztály azokat a billentyűparancsokat tartalmazza, amelyeket a Selenium használhat a billentyűzet szimulálására.
	- A \texttt{By} osztály segítségével különböző módokon kereshetünk elemeket az oldalon (pl. XPath, CSS selector, class név, stb.).
	
	A következő sorban a Chrome böngészőt inicializáljuk, és beállítjuk a szükséges opciókat:
	
	\begin{lstlisting}
		options = webdriver.ChromeOptions()
		driver = webdriver.Chrome(options=options)
	\end{lstlisting}
	
	A \texttt{ChromeOptions} lehetővé teszi a böngésző konfigurálását, például a fej nélküli mód engedélyezését. Az \texttt{options=options} paraméterrel a böngészőt a megadott beállításokkal indítjuk.
	
	\subsubsection{Böngésző méretének beállítása}
	
	A következő sorban a böngésző ablakának méretét 1920x1080 képpontra állítjuk:
	
	\begin{lstlisting}
		driver.set_window_size(1920, 1080)
	\end{lstlisting}
	
	Ez a lépés biztosítja, hogy a teszt minden környezetben megfelelően fusson, függetlenül a képernyő felbontásától.
	
	\subsubsection{Weboldal megnyitása}
	
	A \texttt{driver.get()} metódus segítségével betöltjük a kívánt weboldalt:
	
	\begin{lstlisting}
		driver.get("https://esemenyrendezo.onrender.com/login")
	\end{lstlisting}
	
	Ez a sor a weboldal bejelentkezési oldalát nyitja meg, amelyet a tesztelni kívánunk.
	
	\subsubsection{Email cím és jelszó bevitele}
	
	A következő sorokban a bejelentkezési űrlap mezőit célozzuk meg:
	
	\begin{lstlisting}
		driver.find_element(By.XPATH, "//input[@type='email']").send_keys('TothB')
	\end{lstlisting}
	
	Itt a \texttt{find\_element} metódust használjuk, hogy megtaláljuk az email cím beviteli mezőt. Az \texttt{XPATH} segítségével keresünk rá, és a \texttt{send\_keys()} metódus segítségével beírjuk a tesztelni kívánt értéket.
	
	Hasonló módon a jelszót is beírjuk:
	
	\begin{lstlisting}
		driver.find_element(By.XPATH, "//input[@type='password']").send_keys('a')
	\end{lstlisting}
	
	Ezután két képernyőképet készítünk az aktuális állapotok rögzítésére:
	
	\begin{lstlisting}
		driver.save_screenshot("screenshot1.png")
		driver.save_screenshot("screenshot2.png")
	\end{lstlisting}
	
	\subsubsection{Bejelentkezés és további képernyőképek}
	
	A bejelentkezéshez a következő sorokat használjuk:
	
	\begin{lstlisting}
		driver.find_element(By.CLASS_NAME, "login-btn").click()
	\end{lstlisting}
	
	Ez a sor rákattint a "login-btn" osztályú gombra. Miután a bejelentkezés sikeresen megtörtént, újabb képernyőképet készítünk:
	
	\begin{lstlisting}
		driver.save_screenshot("screenshot3.png")
	\end{lstlisting}
	
	A teszt végén a következő sor biztosítja, hogy a böngésző bezáruljon:
	
	\begin{lstlisting}
		driver.quit()
	\end{lstlisting}
	
\end{document}
